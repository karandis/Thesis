\chapter{Appendix for Chapter 5}

\section{Comments on individual objects}\label{apdx:comments_WNL}
\textbf{WR 108:} According to the GCWR, the spectra of WR 108 suffers from a dilution of emission lines, as well as absorption present in the spectrum. While it is quite common for WNL stars to show absorption, the dilution of emission lines is often attributed to a companion star. We measured a \DelRV{} of ${\sim}16\,$\kms{} over 1140 days, and classified WR 108 as a single star. 

\textbf{WR 120:} The GCWR classifies WR 120 as a single WN7 star. The few absorption lines in the spectra exhibit moderate line-profile variability. However, the S/N of the data in the extreme blue regions makes it challenging to identify if this is simply due to intrinsic variability or due to Doppler motion. With 7 epochs spanning 385 days, we find a \DelRV{} of 40.5\,\kms{} using \niii{}, and classified it as a single star. 

\textbf{WR 123:} WR 123 is notorious for its strong photometric and spectral line-profile variability \citep{marchenko_time-frequency_1998}. It was found to have a periodicity of ${\sim}10\,$hours using photometric and spectroscopic monitoring \citep{lefevre_oscillations_2005,dorfi_most_2006,chene_10-h_2011}. Similar to WR 120, we find the absorption lines in the spectra to vary quite significantly. By monitoring \NVblue{} for eight epochs over 387 days, we measured a \DelRV{} of 57.1\,\kms{} and classified it as a binary system. The sampling and number of spectra are too sparse to find any significant periodicity, and so further monitoring is required to verify its binary status.

\textbf{WR 124:} The GCWR classifies WR 124 as a SB1? with no dilution of emission lines. It is surrounded by a spectacular nebula M1-67 and is the fastest runaway in the Galaxy \citep{moffat_fastest_1982}. Given its 2.4\,d and 2.7\,d periodicity found \citep{moffat_fastest_1982,moffat_photometric_1986}, it was considered to be a binary with a neutron star. \citet{toala_apparent_2018} could not rule out the presence of an embedded neutron star when studying the system with X-rays. If this were the case, the RV amplitude of the WR star would be below our threshold. In any case, we measured \DelRV{} to be 19.3\,\kms{} with 10 epochs over 577 days, and classified WR 124 as a single star.  

\textbf{WR 134:} Similar to WR 123 and WR 124, WR 134 is classified as a SB1? with no dilution of emission lines. Its binary status has been widely debated in the literature, and the observed periodicity has been attributed to a low-mass companion \citep{marchenko_time-frequency_1998,rustamov_spectral_2012}. However, \citet{aldoretta_extensive_2016} demonstrated that large-scale structures in the wind, particularly in \heii{} $\lambda 5412$, exist with a period of 2.255\,$\pm$\,0.008\,d. We obtained 15 epochs over 2901 days and measured \DelRV{} to be 29.6\,\kms{} using \NVred{} and classified it as a single star.

\textbf{WR 136:}  WR 136 is surrounded by a ring nebula (NGC 6888), and is classified as a SB1? with no dilution of emission lines in the GCWR. A period of 4.554\,d was found using line-profile variations \citep{koenigsberger_spectral_1980} and RV measurements \citep{aslanov_hd_1981}. However, this period was not verified later with polarisation \citep{robert_polarization_1989} and photometry \citep{moffat_photometric_1986}. A neutron star companion was suggested as a companion, which would again result in small RV variations of the WR star. With 39 epochs over a time baseline of 2907 days, we measured a \DelRV{} of 15.4\,\kms{} using \heii{} lines and classified it as a single star. 

\textbf{WR 148:} WR 148 is one of the few WN8 stars in the Galaxy that is confirmed to be in a binary system. The GCWR classifies it as a SB1 system, as it has a short period of 4.3\,d and was thought to have a compact companion \citep{drissen_spectroscopic_1986,bracher_wolf-rayet_1979,marchenko_wolf-rayet_1996}. However, \citet{munoz_wr_2017} found a period of 4.317336\,$\pm$\,0.000026 and were able to disentangle the spectra to find a O4-6 companion. \citet{hamann_galactic_2019} found a luminosity ($\log L$\,$(L_\odot)$) of 6.2, indicating its potential to be a main-sequence star. However, they also reported that 15\% of the flux is contributed by its companion, so further analysis taking this into account is required. We obtained 20 epochs over 1161 days and measured RVs using \NIVred{}. We found WR 148 to have a \DelRV{} of 171.8\,\kms{} and classified it as a binary system. 

\textbf{WR 153:} The GCWR classifies WR 153 to be a quadruple system (WN6o/CE+O3-6\,+\,B0:I+B1:V-III) where the WR binary has a period of 6.6887 days and a circular orbit \citep{massey_spectroscopic_1981,demers_quadruple_2002}. With 49 epochs over 2858 days, we measured a \DelRV{} of 576.6\,\kms{} using \NVblue{} and classified it as a binary system. 

\textbf{WR 155:} The GCWR classifies WR 155 as a SB2 (WN6o\,+\,O9II-Ib) system. It is a binary with the shortest known period \citep[1.6412436\,d,][]{moffat_photometric_1986,marchenko_wind-wind_1995}. \citet{drissen_polarimetic_1986} studied the system with polarimetry and found masses of 42 and 30\,\Msun{} for the WN and O stars respectively. We obtained 85 epochs over 3075 days, measured \DelRV{} to be 609.5\,\kms{} using \NVred{} and $\lambda 4058$ and classified it as a binary system. 

\textbf{WR 156:} WR 156 is another WN8 star with large photometric and spectral line-profile variability, with its emission lines diluted. Photometric studies have not found any conclusive periodicity \citep{moffat_photometric_1986,marchenko_wolf-rayet_1998}. It is reported to have a luminosity of 6.01 \citep{hamann_galactic_2019}, which questions its status as a massive WNL versus a classical WNL star. We observed WR 156 over 797 days and obtained 18 epochs. Using \NIVred{} and $\lambda 5737$ lines, we measured a \DelRV{} of 16.1\,\kms{} and classified it as a single star.

\textbf{WR 158:} The GCWR classifies WR 158 as a WN7h star with its emission lines diluted. \citet{andrillat_surprising_1992} discovered the presence of O\,{\sc i} $\lambda 8446$ in its spectrum, hypothesising it being produced in the slowly expanding cool shocked region located outside the ionisation front of the bubble created by the WR wind, or in a binary with a Be companion. \citet{hamann_galactic_2019} reported a luminosity ($\log L$\,$(L_\odot)$) of 6.06, which could potentially indicate its status as a main-sequence WNL star. We obtained 13 spectra over 1031 days, measured a \DelRV{} of 15.0\,\kms{} using \NIVred{} and classified it as a single star.

\section{Relative radial velocity measurements}\label{apdx:rv_measurements_wnl}

Relative RVs for the 11 WNLs in our sample. The reference epoch has a RV of 0.0\,\kms{}. We refrain from providing absolute RV measurements as this is highly method dependent, especially for WR stars. The Barycentric Julian Date (BJD) is given as the middle of the exposure. The average S/N is given in Table\,\ref{tab:time_coverage_spec}. Along with the measured RVs, we have indicated the measurement error, that is, the statistical uncertainty $\sigma_p$ (Eq. \ref{eq:errorCCF}). 

\begin{table}[h!]
    \centering
    \caption{Journal of HERMES observations for WR 108. Mask used: \niii{} $\lambda \lambda 4634, 4641$ and $\lambda \lambda 5321,5327$}
    \begin{tabular}{ccc} \hline \hline
        BJD $-$ 2450000 (d) & Relative RV (\kms) & $\sigma_p$ (\kms) \\ \hline
        7880.6447 & 6.7 & 0.4 \\
        7908.5761 & $-$8.6 & 0.4 \\
        7910.5874 & $-$8.8 & 0.9 \\
        8226.7346 & 0.0 & 0.3 \\
        8249.6662 & $-$1.9 & 0.3 \\
        8255.6139 & $-$3.8 & 0.3 \\
        8647.6343 & $-$8.0 & 0.4 \\
        8651.5446 & $-$4.2 & 0.5 \\
        8671.5345 & $-$9.0 & 0.4 \\
        9002.5960 & 2.1 & 0.6 \\
        9020.6068 & $-$3.3 & 0.4 \\   \hline
    \end{tabular}
    \label{tab:WR108}
\end{table}
\begin{table}[h!]
    \centering
    \caption{Journal of HERMES observations for WR 120. Mask used: \NIII{}}
    \begin{tabular}{ccc} \hline \hline
        BJD $-$ 2450000 (d) & Relative RV (\kms) & $\sigma_p$ (\kms) \\ \hline
        9000.5647 & 8.2 & 1.4 \\
        9000.5754 & 9.7 & 1.8 \\
        9029.5217 & 0.0 & 1.3 \\
        9343.5826 & $-$9.1 & 2.1 \\
        9352.6879 & $-$30.8 & 3.3 \\
        9364.6366 & 2.2 & 1.6 \\
        9385.6108 & $-$21.1 & 1.3 \\   \hline
    \end{tabular}
    \label{tab:WR120}
\end{table}
\begin{table}[h!]
    \centering
    \caption{Journal of HERMES observations for WR 123. Mask used: \NVblue{}.}
    \begin{tabular}{ccc} \hline \hline
        BJD $-$ 2450000 (d) & Relative RV (\kms) & $\sigma_p$ (\kms) \\ \hline
        9009.6572 & $-$41.2 & 4.7 \\
        9041.4644 & $-$5.2 & 6.3 \\
        9078.5129 & 0.0 & 4.9 \\
        9150.3408 & 15.9 & 9.0 \\
        9341.6353 & $-$27.3 & 7.1 \\
        9370.6808 & $-$20.5 & 7.7 \\
        9382.6264 & $-$13.3 & 5.4 \\
        9396.5899 & $-$34.3 & 4.8 \\   \hline
    \end{tabular}
    \label{tab:WR123}
\end{table}
\begin{table}[h!]
    \centering
    \caption{Journal of HERMES observations for WR 124. Mask used: \NIVred{}.}
    \begin{tabular}{ccc} \hline \hline
        BJD $-$ 2450000 (d) & Relative RV (\kms) & $\sigma_p$ (\kms) \\ \hline
        8779.3312 & 2.9 & 4.3 \\ 
        8909.7390 & $-$6.1 & 2.5 \\
        9007.5659 & 0.0 & 1.5 \\
        9024.4541 & 7.3 & 1.3 \\
        9341.7066 & 1.4 & 2.0 \\
        9342.6831 & 12.2 & 1.5 \\
        9345.5820 & 11.4 & 1.7 \\
        9346.6531 & 4.1 & 1.6 \\
        9355.6618 & 13.2 & 1.9 \\
        9356.6234 & 1.0 & 1.5 \\   \hline
    \end{tabular}
    \label{tab:WR124}
\end{table}
\begin{table}[h!]
    \centering
    \caption{Journal of HERMES observations for WR 134. Mask used: \NVred{}.}
    \begin{tabular}{ccc} \hline \hline
        BJD $-$ 2450000 (d) & Relative RV (\kms) & $\sigma_p$ (\kms) \\ \hline
        6119.7082 & $-$17.4 & 1.6 \\
        6889.4453 & 8.5 & 1.0 \\
        7902.7067 & $-$16.2 & 1.8 \\
        7919.7115 & $-$6.3 & 2.6 \\
        7938.7064 & $-$16.6 & 1.8 \\
        8195.7332 & 11.5 & 2.4 \\
        8206.6993 & $-$18.1 & 2.1 \\
        8262.7226 & $-$9.9 & 2.0 \\
        8276.6976 & 0.0 & 1.5 \\
        8625.7223 & $-$3.2 & 1.9 \\
        8652.6205 & $-$15.6 & 2.0 \\
        8663.7183 & $-$7.0 & 2.1 \\
        8779.4245 & 3.8 & 2.8 \\
        9008.7157 & 3.2 & 2.3 \\
        9020.7094 & $-$8.6 & 2.0 \\    \hline
    \end{tabular}
    \label{tab:WR134}
\end{table}
\begin{table}[h!]
    \centering
    \caption{Journal of HERMES observations for WR 136. Mask used: \heii{} $\lambda 4201$, $\lambda 4542$, $\lambda 4687$, $\lambda 5412$, $\lambda 6312$ and $\lambda 6685$.}
    \begin{tabular}{ccc} \hline \hline
        BJD $-$ 2450000 (d) & Relative RV (\kms) & $\sigma_p$ (\kms) \\ \hline
        6113.4883 & $-$2.9 & 0.6 \\
        7899.7208 & $-$3.2 & 0.6 \\
        7915.7252 & $-$4.7 & 0.8 \\
        7919.6966 & $-$2.4 & 1.2 \\
        7988.5162 & $-$2.3 & 0.5 \\
        8195.7629 & 1.2 & 0.8 \\
        8206.7539 & 2.4 & 0.8 \\
        8262.6675 & $-$2.7 & 0.5 \\
        8276.7222 & 0.1 & 0.6 \\
        8308.5859 & $-$6.8 & 0.6 \\
        8608.7275 & 5.6 & 0.6 \\
        8652.6254 & 2.1 & 0.5 \\
        8680.5815 & 0.8 & 0.5 \\
        8700.6889 & 3.2 & 0.4 \\
        8705.5711 & $-$0.4 & 0.4 \\
        8706.5963 & 1.0 & 0.4 \\
        8707.5559 & 4.7 & 0.5 \\
        8708.4991 & 4.7 & 0.4 \\
        8709.4306 & $-$2.9 & 0.4 \\
        8710.5613 & $-$2.2 & 0.4 \\
        8712.5417 & $-$3.3 & 0.4 \\
        8713.5110 & $-$1.9 & 0.5 \\
        8714.4924 & $-$1.1 & 0.4 \\
        8715.5051 & 0.2 & 0.5 \\
        8716.5422 & $-$9.9 & 0.3 \\
        8717.5973 & $-$0.5 & 1.1 \\
        8718.4901 & $-$2.0 & 0.4 \\
        8719.4433 & $-$0.2 & 0.4 \\
        8720.4152 & 0.9 & 0.5 \\
        8721.5437 & $-$3.9 & 0.5 \\
        8729.5184 & 2.0 & 0.3 \\
        8766.4506 & 1.3 & 0.2 \\
        8767.4729 & 1.0 & 0.3 \\
        8771.3722 & 1.4 & 0.2 \\
        8772.4400 & 2.7 & 0.2 \\
        8773.3992 & 0.0 & 0.3 \\
        8779.4334 & $-$0.7 & 0.5 \\ 
        9009.5365 & 1.2 & 0.5 \\
        9020.6861 & $-$2.0 & 0.6 \\   \hline
    \end{tabular}
    \label{tab:WR136}
\end{table}
\begin{table}[h!]
    \centering
    \caption{Journal of HERMES observations for WR 148. Mask used: \NIVred{} and $\lambda 4058$.}
    \begin{tabular}{ccc} \hline \hline
        BJD $-$ 2450000 (d) & Relative RV (\kms) & $\sigma_p$ (\kms) \\ \hline
        7912.6900 & 51.0 & 1.2 \\
        7934.7125 & $-$22.9 & 1.3 \\
        7952.6962 & $-$74.7 & 1.0 \\
        8203.7536 & $-$60.6 & 0.7 \\
        8262.6954 & 0.0 & 0.6 \\
        8332.6787 & $-$80.9 & 0.6 \\
        8684.5886 & 90.9 & 0.8 \\
        8757.4993 & 65.7 & 0.8 \\
        8775.4903 & 87.1 & 1.0 \\
        8776.4943 & 12.7 & 1.2 \\
        8778.4304 & 4.2 & 3.7 \\
        8779.3554 & 89.3 & 4.8 \\
        8780.4683 & 52.8 & 1.5 \\
        8781.3212 & $-$51.7 & 1.4 \\
        8791.3731 & 5.0 & 1.6 \\
        8792.3957 & 82.5 & 1.7 \\
        8796.4216 & 73.3 & 1.2 \\ 
        8798.3922 & $-$46.3 & 1.9 \\
        9029.6207 & 63.2 & 1.3 \\
        9073.5716 & 59.8 & 1.1 \\   \hline
    \end{tabular}
    \label{tab:WR148}
\end{table}
\begin{table}[h!]
    \centering
    \caption{Journal of HERMES observations for WR 153. Mask used: \NVblue{}.}
    \begin{tabular}{ccc} \hline \hline
        BJD $-$ 2450000 (d) & Relative RV (\kms) & $\sigma_p$ (\kms) \\ \hline
        6305.3277 & $-$11.8 & 3.0 \\
        6889.5378 & 41.7 & 3.2 \\
        6889.5592 & 27.0 & 3.7 \\
        6889.7109 & $-$19.2 & 3.8 \\
        6889.7324 & 0.0 & 5.0 \\
        6890.4169 & $-$151.0 & 4.1 \\
        6890.4383 & $-$152.5 & 4.5 \\
        6890.6822 & $-$223.8 & 3.3 \\
        6890.7036 & $-$211.2 & 4.1 \\
        6892.4550 & $-$336.0 & 2.9 \\
        6892.4764 & $-$372.0 & 2.9 \\
        6892.7059 & $-$293.0 & 2.2 \\
        6892.7273 & $-$291.0 & 3.0 \\
        6894.3935 & 49.1 & 3.1 \\
        6894.4149 & 58.0 & 3.4 \\
        6894.5856 & 79.3 & 2.6 \\
        6894.6070 & 93.3 & 4.0 \\
        6896.4275 & 3.6 & 2.6 \\
        6896.4489 & 0.0 & 2.6 \\
        6896.6675 & $-$71.7 & 5.7 \\
        6896.6855 & $-$49.4 & 3.5 \\
        6903.4817 & $-$64.8 & 3.0 \\
        6903.5031 & $-$82.6 & 2.9 \\
        6903.6291 & $-$123.0 & 3.1 \\
        6903.6505 & $-$123.6 & 2.7 \\
        6946.3828 & $-$278.5 & 2.3 \\
        6946.4121 & $-$279.7 & 1.9 \\
        6946.5195 & $-$233.3 & 3.4 \\
        6946.5435 & $-$229.1 & 4.1 \\
        7952.7273 & 113.9 & 3.9 \\
        7958.7316 & 179.4 & 8.9 \\
        7963.6529 & $-$101.2 & 5.5 \\
        8102.3196 & $-$318.9 & 3.8 \\
        8198.7456 & 78.9 & 5.2 \\
        8267.7027 & 16.0 & 5.2 \\
        8310.6395 & $-$287.8 & 3.7 \\
        8319.6697 & 151.2 & 4.4 \\
        8660.7183 & 130.3 & 4.1 \\
        8754.4247 & 130.2 & 3.2 \\
        8760.5219 & 27.5 & 3.6 \\
        8775.5577 & 72.1 & 2.6 \\
        8779.4643 & $-$220.4 & 3.1 \\
        8796.4808 & $-$122.2 & 3.9 \\
        8798.4221 & $-$397.2 & 0.0 \\
        8799.4985 & $-$238.5 & 2.7 \\
        8800.4418 & 11.9 & 6.5 \\
        8819.3970 & $-$255.4 & 3.8 \\
        8820.3165 & $-$40.2 & 5.7 \\
        8858.3385 & $-$104.5 & 0.0 \\
        9010.6960 & $-$153.8 & 2.4 \\
        9044.6388 & $-$262.1 & 2.4 \\
        9073.6140 & $-$223.5 & 2.5 \\
        9074.5856 & $-$47.6 & 3.7 \\
        9075.6012 & 94.5 & 3.2 \\
        9076.5776 & 50.8 & 2.3 \\
        9077.5882 & $-$145.7 & 2.7 \\
        9078.6008 & $-$325.5 & 1.3 \\
        9079.5884 & $-$351.2 & 2.1 \\
        \hline
    \end{tabular}
    \label{tab:WR153}
\end{table}
\begin{table}[h!]
    \centering
    \caption{Continued for WR 153.}
    \begin{tabular}{ccc} \hline \hline
        BJD $-$ 2450000 (d) & Relative RV (\kms) & $\sigma_p$ (\kms) \\ \hline
        9080.5761 & $-$185.6 & 1.9 \\
        9146.3619 & $-$321.9 & 3.3 \\
        9147.4905 & $-$160.5 & 4.9 \\
        9148.3736 & 28.2 & 3.5 \\
        9151.4101 & $-$202.9 & 2.3 \\
        9152.3290 & $-$320.1 & 1.6 \\
        9153.4017 & $-$312.2 & 3.1 \\
        9154.4319 & $-$97.8 & 3.4 \\
        9155.5509 & 79.1 & 5.4 \\ 
        9156.5313 & 114.1 & 2.9 \\
        9157.3331 & $-$10.4 & 3.7 \\
        9163.4296 & 74.1 & 21.3 \\   \hline
    \end{tabular}
    \label{tab:WR153_2}
\end{table}
\begin{table}[h!]
    \centering
    \caption{Journal of HERMES observations for WR 155. Mask used: \NVred{} and $\lambda 4058$.}
    \begin{tabular}{ccc} \hline \hline
        BJD $-$ 2450000 (d) & Relative RV (\kms) & $\sigma_p$ (\kms) \\ \hline
        6132.6769 & 316.0 & 0.9 \\ 
        6134.5244 & 324.1 & 0.9 \\ 
        6135.6394 & $-$14.1 & 0.9 \\ 
        6209.4378 & $-$89.5 & 0.9 \\ 
        6210.3816 & 60.3 & 1.0 \\ 
        6210.4044 & 24.5 & 1.1 \\ 
        6210.5143 & $-$118.3 & 1.0 \\ 
        6210.5288 & $-$133.4 & 0.9 \\ 
        6211.4146 & 290.0 & 0.7 \\ 
        6212.4061 & $-$247.2 & 0.9 \\ 
        6213.5000 & 253.7 & 1.1 \\ 
        6214.4316 & 0.0 & 1.0 \\ 
        6215.5349 & $-$203.8 & 0.6 \\ 
        7948.6674 & $-$197.8 & 0.8 \\ 
        7956.6991 & $-$23.8 & 1.2 \\ 
        7962.7296 & 354.7 & 1.2 \\ 
        8091.4423 & $-$194.7 & 1.5 \\ 
        8102.3489 & 342.7 & 1.2 \\ 
        8132.3478 & $-$85.3 & 1.3 \\ 
        8335.5487 & 278.6 & 1.8 \\ 
        8336.6230 & 98.9 & 1.7 \\ 
        8668.7252 & 267.6 & 1.3 \\ 
        8721.6561 & $-$171.8 & 1.2 \\ 
        8733.5426 & $-$160.8 & 1.6 \\ 
        8775.3155 & 298.6 & 1.1 \\ 
        8775.6162 & 48.7 & 1.6 \\ 
        8776.5293 & 159.5 & 0.8 \\ 
        8776.5871 & 217.1 & 0.8 \\ 
        8778.4172 & 335.2 & 1.6 \\ 
        8778.5538 & 314.4 & 1.3 \\ 
        8779.3193 & $-$235.6 & 3.7 \\ 
        8779.4735 & $-$188.6 & 2.9 \\ 
        8794.3532 & $-$102.8 & 5.0 \\ 
        8794.3736 & $-$77.8 & 2.7 \\ 
        8796.3591 & 287.6 & 1.0 \\ 
        8796.4956 & 333.0 & 1.3 \\ 
        8800.3673 & $-$105.3 & 1.2 \\ 
        8819.3904 & 331.3 & 1.2 \\ 
        8819.4572 & 350.2 & 1.4 \\ 
        8820.3269 & $-$239.6 & 1.4 \\ 
        8858.3542 & $-$100.5 & 1.6 \\ 
        8859.3150 & 82.0 & 1.8 \\ 
        9009.7119 & 288.0 & 1.1 \\ 
        9041.5810 & $-$61.4 & 2.2 \\ 
        9042.5025 & 259.4 & 1.5 \\ 
        9042.6246 & 334.2 & 1.2 \\ 
        9044.6499 & 206.8 & 1.1 \\ 
        9076.4782 & $-$201.3 & 2.6 \\ 
        9077.5154 & 161.5 & 1.8 \\ 
        9077.6831 & $-$64.1 & 1.2 \\ 
        9078.4845 & 134.0 & 1.3 \\ 
        9080.5906 & 312.5 & 1.1 \\ 
        9082.5059 & 76.7 & 1.2 \\ 
        9082.7455 & $-$183.4 & 3.0 \\ 
        9083.3961 & 118.0 & 1.6 \\ 
        9083.5871 & 297.5 & 1.2 \\ 
        9088.5333 & 312.8 & 1.7 \\ 
        9088.6721 & 332.2 & 1.6 \\ 
        \hline
    \end{tabular}
    \label{tab:WR155}
\end{table}
\begin{table}[h!]
    \centering
    \caption{Continued for WR 155.}
    \begin{tabular}{ccc} \hline \hline
        BJD $-$ 2450000 (d) & Relative RV (\kms) & $\sigma_p$ (\kms) \\ \hline
        9089.6076 & $-$211.0 & 4.3 \\ 
        9089.6304 & $-$200.2 & 1.2 \\ 
        9098.5925 & 323.5 & 1.3 \\ 
        9098.7359 & 263.4 & 1.3 \\ 
        9099.5498 & $-$154.5 & 2.8 \\ 
        9146.3472 & 246.8 & 1.7 \\ 
        9146.5890 & $-$30.0 & 2.2 \\ 
        9147.4761 & 187.5 & 1.3 \\ 
        9147.5661 & 277.5 & 1.6 \\ 
        9148.3213 & $-$139.8 & 3.2 \\ 
        9148.5401 & $-$254.9 & 4.1 \\ 
        9152.3438 & 136.5 & 1.3 \\ 
        9152.5696 & 338.3 & 1.8 \\ 
        9153.4166 & $-$244.0 & 3.3 \\ 
        9153.5375 & 270.5 & 12.1 \\ 
        9154.4476 & 306.5 & 1.9 \\ 
        9155.3228 & $-$186.9 & 2.4 \\ 
        9155.5357 & 11.6 & 1.4 \\ 
        9157.3185 & 195.3 & 1.2 \\ 
        9157.3482 & 221.8 & 1.1 \\ 
        9160.4016 & $-$34.4 & 1.0 \\ 
        9162.3752 & 311.7 & 1.1 \\ 
        9196.4044 & $-$148.5 & 2.4 \\ 
        9206.3166 & $-$56.8 & 2.1 \\ 
        9206.4406 & 61.1 & 1.7 \\ 
        9207.3304 & $-$42.3 & 2.1 \\ 
        9207.4586 & $-$172.5 & 11.4 \\    \hline
    \end{tabular}
    \label{tab:WR155_2}
\end{table}
\begin{table}[h!]
    \centering
    \caption{Journal of HERMES observations for WR 156. Mask used: \NIVred{} and $\lambda 5737$.}
    \begin{tabular}{ccc} \hline \hline
        BJD $-$ 2450000 (d) & Relative RV (\kms) & $\sigma_p$ (\kms) \\ \hline
        8336.6487 & 5.4 & 1.2 \\
        8754.4051 & $-$2.9 & 1.1 \\
        8761.5931 & 0.0 & 1.1 \\
        8775.4585 & $-$3.5 & 1.0 \\
        8786.4646 & $-$9.7 & 1.2 \\
        8798.5261 & $-$6.0 & 1.7 \\
        8798.5368 & $-$4.7 & 2.1 \\
        8821.3559 & $-$6.3 & 1.2 \\
        8826.4376 & $-$8.9 & 1.2 \\
        9073.6574 & 0.3 & 1.0 \\
        9079.6211 & $-$6.5 & 1.0 \\
        9081.5919 & $-$0.2 & 1.0 \\
        9106.6019 & $-$3.9 & 1.2 \\ 
        9120.5904 & $-$2.5 & 1.3 \\
        9124.4972 & $-$8.8 & 1.2 \\
        9127.5059 & $-$8.6 & 1.3 \\
        9132.5919 & 6.4 & 1.1 \\
        9133.5744 & $-$2.8 & 1.1 \\   \hline
    \end{tabular}
    \label{tab:WR156}
\end{table}
\begin{table}[h!]
    \centering
    \caption{Journal of HERMES observations for WR 158. Mask used: \NIVred{}.}
    \begin{tabular}{ccc} \hline \hline
        BJD $-$ 2450000 (d) & Relative RV (\kms) & $\sigma_p$ (\kms) \\ \hline
        8091.4727 & $-$2.7 & 0.4 \\ 
        8336.7140 & 3.1 & 0.4 \\ 
        8736.6307 & $-$1.9 & 0.4 \\ 
        8737.5247 & 0.0 & 0.3 \\ 
        8775.3857 & $-$1.0 & 0.4 \\ 
        8798.4824 & $-$3.5 & 0.5 \\ 
        8821.4275 & $-$2.2 & 0.3 \\ 
        8875.3741 & 6.5 & 0.4 \\ 
        9075.6776 & $-$4.5 & 0.4 \\ 
        9089.6552 & $-$8.5 & 0.5 \\ 
        9097.6090 & 3.6 & 0.6 \\ 
        9106.6458 & 0.6 & 0.5 \\ 
        9122.4711 & 4.6 & 0.4 \\    \hline
    \end{tabular}
    \label{tab:WR158}
\end{table}