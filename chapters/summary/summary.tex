\chapter{Summary and future work}\label{ch:summary}

\section{Summary}
In this thesis, we investigated the intrinsic multiplicity properties, namely the binary fraction and properties of the orbital-period distribution for the Galactic WN and WC population. We did this by undertaking a spectroscopic monitoring campaign over about five years for a magnitude-limited sample of northern, Galactic WR stars using the HERMES spectrograph mounted on the Mercator telescope on the island of La Palma. For the HERMES spectrograph, we developed a method to correct the pipeline-produced scientific spectra for the instrumental response. Furthermore, we measured RVs for the objects in our sample using a cross-correlation method allowing us to properly quantify the uncertainty on the measurements. We further investigated the effects of wind-variability on RV measurements for different spectral sub-types of WR stars. We quantified the observed binary fraction for the WC, WNE and WNL objects in our sample by measuring their RV variability amplitudes. Finally, we used Monte-Carlo simulations with a Bayesian framework to correct for observational biases and constrain the intrinsic multiplicity properties of the parent population.

In Chapter \ref{ch:data_reduction}, we developed the instrumental response correction for the HERMES spectrograph as an improvement to the current data-reduction pipeline. We first used the publicly-available code Molecfit to correct the calibration spectra for telluric contamination. Following this we divided it by a model SED from GSSP and applied a median filter to get the instrumental response. We corrected all of the WR spectra for telluric contamination with Molecfit and applied the correction for the instrumental response. The shape of the resulting spectra then resembled the spectral energy distribution. We used scaled continuum models from the model-atmosphere code PoWR and applied an interstellar reddening to then normalise the spectra. In this way the normalised spectra, and hence RV measurements, were not affected by human systematics.

In Chapter \ref{ch:wc}, we analysed spectroscopic time series of 12 Galactic WC stars with at least six epochs obtained over 2-8 years. We measured relative RVs using cross-correlation with a maximum log-likelihood framework in order to derive accurate uncertainties on our measurements. Furthermore, we measured RVs using high-cadence data for WR 137, a known long-period binary, away from periastron to quantify the effect of intrinsic variability. Using C\,{\sc iv} lines, we found an upper limit of 3\,\kms{} due to variability in the wind. After measuring the values of \DelRV{},
we found 7 out of 12 stars showing RV variations more than 10\,\kms{} and classified them as candidate binaries, resulting in an observed binary fraction of $0.58\pm0.14$. Finally, we showed using MC simulations that the intrinsic multiplicity of Galactic WC stars is at least 0.72 with periods up to $10^5$\,d.

In Chapter \ref{ch:wne}, we analysed the 16 WNE stars in our sample with at least six epochs obtained over 3-8 years. We found that the spectra of WNE stars have larger line-profile variability than WC stars on average. Based on the high-cadence data of the long-period binary WR 138, we were able to quantify the peak-to-peak variability using \nv{} lines to be 15\,\kms{}. Furthermore, we used lines of \nv{} to measure RVs for most of the objects in our sample and derived \DelRV{} values for them. We chose a threshold of 50\,\kms{} and found an observed binary fraction of $0.44\,\pm0.12$. We developed a Bayesian framework around MC simulations to correct for observational biases and found that the majority of WNE binaries have orbital periods between 1 and 10 days with a binary fraction of $0.56\substack{+0.20 \\ -0.15}$. Moreover, the derived multiplicity properties are similar to what is found for Galactic O stars and can be explained with our current understanding of binary evolution. We also revisited the 12 WC stars from Chapter \ref{ch:wc} and found that the majority of WC binaries exist with orbital periods between 1000 and 10\,000 days wtih a binary fraction of $0.96\substack{+0.04 \\ -0.22}$. The multiplicity properties of both the WC and WNE populations are consistent with what is observed in the Galaxy, reaffirming that our selected samples are representative of the whole population.

Finally, in Chapter \ref{ch:wnl}, we analysed a sample of 11 WNL stars with at least six epochs obtained over 4-8 years. As observed in the literature, we found that WNL stars have larger line-profile variability than WNE and WC stars. Due to the lack of a long-period binary in our sample, we analysed the apparently single star WR 136 with high-cadence data to study the effects of wind variability on RV measurements. We found an amplitude of 15\,\kms{} by measuring RVs using lines of \heii{}. WR 136 is the only object in our entire sample of 39 WR stars where the variability in \heii{} is less than other lines of higher ionisation, demonstrating once again that every WR star is unique. We adopted a threshold of 50\,\kms{} and found the observed binary fraction to be $\,0.36\,\pm\,0.15$. After correcting for observational biases, we find the intrinsic binary fraction to be $0.42\substack{+0.15 \\ -0.17}$. The properties of the period distribution are consistent with what is found for the WNE sample, allowing us to perform a combined analysis of the WN sample. We found the intrinsic binary fraction to be $0.52\substack{+0.14 \\ -0.12}$ with a majority of binaries having periods less than 10 days. This is in contrast with the Galactic WC population with very long period binaries. According to the Conti scenario, WN stars are the descendants of WC stars and so these binary populations must also be connected. The high-frequency of short-period WN binaries and lack of WC counterparts suggests that these systems avoid evolving into WC binaries. Two possible explanations for this are mergers, or that the WN star does not reach the WC phase in such systems. Similarly, the apparent lack of long-period WN binaries raises the question about the progenitors of long-period WC binaries. As orbital evolution from the WN to the WC phase is governed by mass-loss, this change in the period is irreconcilable. A tempting explanation is that we do not detect long-period WN binaries, their natural progenitors, because low-amplitude RV variations are masked by wind-variability. If this were the case, then the intrinsic binary fraction of the WN population would jump to ${\sim}1.00$, similar to that of the WC population.


% With this in mind, the following questions are of interest to us: what fraction of Galactic WR stars reside in binaries? What are the orbital configurations of these binaries? Do they give us insight into past events of binary interaction from the main-sequence phase (e.g. mergers, ejections)? Is the observed discrepancy in the orbital period distribution between WN and WC stars real, or is it simply a bias? What is the effect of multiplicity on the Conti scenario?



Before proceeding to the future work, let us take a moment to reexamine the questions posed in Sect. \ref{sect:motivation_intro}. We found that the binary fraction of the Galactic WN population is $0.52\substack{+0.14 \\ -0.12}$, while that of the Galactic WC population is $0.96\substack{+0.04 \\ -0.22}$. These populations both have binaries with periods up to $10^5\,$d. The discrepancy in the observed period distributions in the Galaxy seems to be real, since even after correcting for observational biases, we find the majority of WNs to reside in short-period binaries and that of WCs to reside in long-period binaries. If all short-period WNs were to fail to become WCs, then the Conti scenario is strongly affected by multiplicity. However, the role of mergers, particularly further evolution of the merger product, is unclear.

\section{Future work}

\subsection{Expanding the sample}

The work presented in this thesis is the largest homogeneous analysis for multiplicity in Galactic WR stars. Given the extremely short lifetime of the WR phase, expanding such studies to a larger fraction of the 667 Galactic WR stars is crucial to understand the effects of multiplicity on the evolution of these stars. In order to probe long-period binaries, part of the sample presented in this work will be monitored on a long-term basis with HERMES.

\subsubsection{The southern Galactic sample}

The natural follow-up to this study would be to complement it with a sample of southern Galactic WR stars. Given how optically faint most WR stars are, using facilities with the capability to probe fainter $V$-band magnitudes would be a logical next step. To prepare for this, we have already acquired between 3 and 6 epochs of spectra for 63 WR stars in the southern hemisphere using the High Resolution Spectrograph mounted on the  South-African Large Telescope (SALT). A similar analysis can be performed on this data-set, although one point of concern would be the homoegeneity of the normalisation and its impact on RV measurements, since the method developed here cannot be applied to SALT data in a straightforward manner.

\subsubsection{Interferometry to probe longer periods}

One of the potential explanations for the progenitors of long-period WC binaries is the presence of undetected long-period WN binaries. Given the period regime, interferometry is a better-suited technique to hunt for these systems (Sect. \ref{sect:detection_methods}). We have already applied for a large programme at the European Southern Observatory (ESO) to acquire a snapshot of southern Galactic WR stars. This data can be complemented with a similar data-set from the northern hemisphere. For this purpose, we have acquired data of 6 WR stars with the Center for High Angular Resolution Astronomy (CHARA) array. Analysis of these data-sets and correcting them for observational biases is the next step forward.

\subsection{An improved Bayesian analysis}

The Bayesian analysis in this work attempts to find the likelihood of reproducing the objects in our sample in various \DelRV{} bins based on certain input parameters. A first step could be to incorporate the mass-ratio as an additional parameter, but this would currently take too much computational time. Other parameters can also be included in such a way, given an increase in the sample. However, for each object one would ideally calculate the likelihood of observing a particular RV timeseries based on various input parameters. This would provide a deeper insight onto the possible orbital configurations and mass-ratios of the WR binary population.

\subsection{Orbital analysis of WR binaries}

Aside from expanding monitoring surveys, efforts must be made to increase the number of WR binaries with derived orbital solutions. Orbital analysis allows us to infer dynamical masses and radii of the components in a model-independent manner. This is a powerful technique which can be combined with model-atmospheric analysis to lift some of the degeneracy \citep[e.g. in][]{richardson_chara_2016,2021Richardson}. The eccentricity of such binary systems can also be derived, which is a critical quantity when trying to understand formation scenarios and further evolution. The goal of the long-term HERMES monitoring programme is also to sample the known binaries appropriately in phase such that their orbits can be improved.

Lastly, Gaia DR3 and DR4 will detect a large number of astrometric binaries and provide accurate astrometric solutions. For example, almost 200 OB\,+\,BH binaries expected to be detected with Gaia DR3 \citep{2022Janssens}. Therefore it is possible that a few tens of WR binaries are also detected.

In conclusion, the work presented in this thesis attempts to address some of the open questions with respect to the effects of multiplicity on massive star evolution. The results presented in this thesis raise many questions on the evolutionary connection between WN and WC binaries. The upcoming future is promising, with a lot of different perspectives with which the multiplicity of WR stars can be probed.


% -runaways WNL
% -improve bayesian for mass-ratio dist too
% - improve bayesian for predicting rv and not bins
